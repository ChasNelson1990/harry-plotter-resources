\documentclass[a4paper]{article}
\usepackage[cm]{fullpage}
\usepackage{hyperref}
\title{d3.js and plotly Resources}
\begin{document}
    \maketitle
    This file contains resources for learning more about d3.js and plotly.\\

    \section{d3.js}
    \begin{enumerate}
        \item Start with the d3 home pages: \url{https://d3js.org/}.\\
        \item There are hundred of examples on Mike Bostock's pages: \url{http://bl.ocks.org/mbostock}.\\
        \item But some more useful examples can be found here: \url{http://www.verisi.com/resources/d3-tutorial-basic-charts.htm}.\\
        \item And it's well documented, including loads of tutorials: \url{https://github.com/mbostock/d3/wiki/Tutorials}.\\
        \item And if your questions hasn't been answered, try stack overflow: \url{https://stackoverflow.com/questions/tagged/d3.js}.\\
        \item Mike Bostock's blog (\url{https://bost.ocks.org/mike/}) is probably the best to follow but you can find more on the github pages: \url{https://github.com/mbostock/d3/wiki/Tutorials#blogs}.\\
        \\
        \item \textbf{NOTE}: you may need to set up a local webserver on your PC for full functionality; follow the tutorial at \url{http://chimera.labs.oreilly.com/books/1230000000345/ch04.html}.\\
    \end{enumerate}

    \section{plotly}
    \begin{enumerate}
        \item The biggest resource will be the plotly pages: \url{https://plot.ly/}.\\
        \item Discover more about the API for your favourite software: \url{https://plot.ly/api/}.\\
        \item Acess the web app help pages and the community pages here: \url{https://plot.ly/api/}.\\
        \item Or dive in straight to the full function list, with examples, here: \url{https://plot.ly/javascript/plotlyjs-function-reference/}.\\
        \item And, finally, some tutorials here: \url{http://help.plot.ly/tutorials/}.\\
    \end{enumerate}

\end{document}
